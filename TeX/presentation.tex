\documentclass[11pt,aspectratio=1610,dvipsnames]{beamer}
\graphicspath{{figs/}}
\usetheme{default}
\usepackage{DasBeamerPaket}
\usepackage{animate}
\usepackage{lastpage}
\usepackage{tikz}
\usepackage{lmodern}
\setbeamercolor{section in toc}{fg=NavyBlue}
\setbeamercolor{frametitle}{fg=NavyBlue}
\captionsetup[figure]{labelfont=bf}
\captionsetup[table]{labelfont=bf}
\setbeamertemplate{caption}[numbered]
\title[$\Lambda(1405)$]{Experimental studies of the $\Lambda(1405)$}
\subtitle[Seminar physics654]{physics654 -- Seminar on exotic multi-quark states}

\begin{document}
\definecolor{myWhite}{rgb}{1,1,1}

\setbeamertemplate{footline}[text line]{\parbox{0.3\linewidth}{\vspace*{-9pt}\textcolor{white} \insertsection  \hfill} \parbox{0.7\linewidth}{\vspace*{-8pt} \textcolor{white}{\hfill\hspace{-3cm}\insertshorttitle \phantom{ }-- \insertshortsubtitle}  \hfill \textcolor{myWhite}{\insertpagenumber/\pageref{LastPage}}}}

\addtobeamertemplate{footline}{ \makebox[0pt][l]{\hspace{-1cm}
		\raisebox{0cm}[0pt][0pt]{\colorbox{gray!20!black}{\phantom{{\large TEXTTEXTTEXTTEXTTEXTTEXTTEXTTEXTTEXTTEXTTEXTTEXTTEXTTEXTTEXTTEXTTEXTTEXTTEXTTEXTTEXTTEXTTEXTTEXTTEXTTEXTTEXTTEXTTEXTTEXTTEXTTEXTTEXTTEXTTEXTTEXTTEXTTEXTTEXTTEXT}}}}}}

\setbeamercovered{transparent}
\setbeamertemplate{navigation symbols}{}
\setbeamertemplate{frametitle}[default][left,leftskip=0.5cm]
%
\setbeamertemplate{itemize item}{\color{black}$\blacktriangleright$}
\setbeamertemplate{section in toc}[sections numbered]
\captionsetup{font=scriptsize,labelfont=scriptsize}

\AtBeginSection[]
{	
	\definecolor{myWhite}{rgb}{0,0,0}
	\begin{frame}[noframenumbering]
		\frametitle{}
		\addtocounter{page}{-1}
		\tableofcontents[currentsection]
		
	\end{frame}
\definecolor{myWhite}{rgb}{1,1,1}
}


\begin{frame}[plain]
	\setcounter{page}{0}
	\centering
	{\Large \color{MidnightBlue}{Experimental studies of the $\Lambda(1405)$}}\\
	{\href{https://www.youtube.com/watch?v=oHg5SJYRHA0}{physics654 -- Seminar on exotic multi-quark states}}
	

	\vfill

		



			
	\textsc{Jakob Krause}\\
	\scriptsize \href{mailto:krause@hiskp.uni-bonn.de}{\faEnvelope  \hspace*{0.1cm}krause@hiskp.uni-bonn.de} {\color{black}$|$} \href{https://github.com/krausejm}{\faGithub  \hspace*{0.1cm}krausejm}\\
	
	\vspace{.5cm}
	
	Tutor: \textsc{Georg Scheluchin}\\
	 \href{mailto:scheluchin@physik.uni-bonn.de}{\faEnvelope  \hspace*{0.1cm}scheluchin@physik.uni-bonn.de}

	\vspace{0.2cm}
	
	18.06.2021
	 	
 		
\end{frame}
\section*{Motivation}
\begin{frame}{Motivation}
\begin{minipage}{\linewidth}
		\begin{tcolorbox}[colback=black!10,colframe=gray!20!black,title=What is special about the $\Lambda(1405)$?] 
			\begin{itemize}
				\item its mass does not fit well into constituent quark models which do predict baryon masses well for other baryons 
				\item invariant mass distribution (line shape) differs significantly from usual \textsc{Breit-Wigner} shapes
				\item candidate for an exotic multiquark state (bound system of $\overline{K}N$) since its mass lies just below threshold
			
 			\end{itemize}
 		\vspace{.5cm}
 		There are (very) many different theoretical approaches to explain this behavior\\
 		$\to$ There is need for more experimental data!
		\end{tcolorbox}
	{\color{red} some plots/pictures?}
\end{minipage}




	
\end{frame}
%\begin{frame}[plain]
%	\maketitle
%	\setcounter{page}{0}
%\end{frame}
\section*{Table of contents}
\begin{frame}{Table of contents}
	\tableofcontents
\end{frame}
\section{Experimental setup}
\begin{frame}{Continuous Electron Beam Accelerator Facility (CEBAF)}
	\begin{figure}
		\centering
		\includegraphics[width=\linewidth]{setup_big.jpg}
		\caption{CEBAF layout at Jefferson Lab, [\cite{clas}]}
	\end{figure}
\end{frame}
\begin{frame}{CEBAF Large Acceptance Spectrometer (CLAS)}
	\begin{figure}
		\centering
		\includegraphics[width=.7\linewidth]{setup.jpg}
		\caption{CLAS layout at Jefferson Lab, [\cite{clas}]}
	\end{figure}
\end{frame}


\section{Line-shape measurement}


\section{Spin-parity measurement}
\begin{frame}{Theoretical basics}
	The $\Lambda(1405)$ is so far (mostly) assumed to have $J^P=\frac{1}{2}^-$, but this has not been determined experimentally
	\begin{tcolorbox}[colback=black!10,colframe=gray!20!black,title=Measuring spin] 
		\begin{itemize}
			\item consider the strong decay $Y^*\to Y\pi$, with $J^P$ the spin and parity of $Y^*$
			\item the $Y\pi$ angular distribution will only depend on $J$
			\begin{align*}
				I(\theta_Y)&=\text{const.} & J=1/2\\
				I(\theta_Y)&\propto 1+\frac{3(1-2p)}{2p+1}\cos^2\theta_Y& J=3/2,
			\end{align*}
			where $\theta_Y$ is the polar angle of the decay direction of $Y$ in the $Y^*$ rest frame, $p$ describes the fraction of spin projections along the $z$ axis 
			\item uniform decay pattern is best evidence for spin $J=1/2$
			
		\end{itemize}
	\end{tcolorbox}
	\begin{flushright}
		[\cite{spinparity}]
	\end{flushright}
	
\end{frame}
\begin{frame}{Theoretical basics}
	\begin{tcolorbox}[colback=black!10,colframe=gray!20!black,title=Measuring parity] 
		\begin{itemize}
			\item the key to accessing the parity lies in determining the Polarization transfer to the decay product $Y$ which we will denote $\mathbf{Q}$
			\item the angular distribution of $Q$ will only depend on $P$
			\begin{align*}
				\mathbf{Q}(\theta_Y)&=\text{const.} & J^P=1/2^-\\
				Q(\theta_Y)&=-P+2(P\cdot q)q & J^P=1/2^+,
			\end{align*}
			
			\item uniform decay pattern is best evidence for spin $J=1/2$
			
		\end{itemize}
	\end{tcolorbox}
	\begin{flushright}
		[\cite{spinparity}]
	\end{flushright}
	
\end{frame}

\section{Conclusion}








\end{document}